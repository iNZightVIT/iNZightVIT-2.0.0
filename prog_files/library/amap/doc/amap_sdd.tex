\documentclass[twoside,10pt]{article}
\usepackage[latin1]{inputenc}
\usepackage[english,french]{babel}
\usepackage{multicol}
\usepackage{a4wide}
%\usepackage{a5}
%\usepackage{graphics}
\usepackage{fancyhdr}
\usepackage{verbatim}
%\usepackage{hyperref}
\usepackage{lettrine}
%\usepackage{times}
%\usepackage[Glenn]{fncychap}
\usepackage{color} 
%\usepackage{draftcopy}
%\usepackage{SFHeaders}

%\usepackage{hieroglf} % hieroglyphes

\usepackage{geometry}

\bibliographystyle{amsplain}

%dvips  -T 14.85cm,21cm -o amap_sdd.ps amap_sdd.dvi; psbook amap_sdd.ps tmp.ps ;  psnup -2 -pa4 -Pa5 tmp.ps tmpa4.ps; pstops '2:0U(21cm,29.7cm),1' tmpa4.ps amap_sdd.ps

%\geometry{verbose,paperwidth=148.5mm,paperheight=210mm,
%  tmargin=15mm,bmargin=15mm ,lmargin=25mm,rmargin=20mm}
\usepackage{graphicx} 
\DeclareGraphicsExtensions{.eps} 

\pagestyle{fancy}
\renewcommand{\headrulewidth}{0pt}
\fancyhead{}
\cfoot{$\thepage$}

\title{\bf Amap Software Design Description \\(SDD)}
%\date{}
\author{Antoine Lucas\thanks{e-mail: antoinelucas@gmail.com}\\[0.2cm]
\normalsize Brest
}

\newcounter{numeroreq}

	
\newcommand{\req}[1]{\stepcounter{numeroreq} \small{ \em \noindent [BEGIN REQUIREMENT AMAP\_SDD\_\arabic{numeroreq}] }\\ \noindent \rm \normalsize #1 \\ \small{ \em \noindent [END REQUIREMENT] \\ } }

\begin{document}


\maketitle

\sloppy


%% \begin{table}[htb]
%% \centering
%% \begin{tabular}{cc}
%% \hline
%% \bf Name & \bf e-mail \\
%% \hline
%% \hline
%% Antoine Lucas & antoinelucas@gmail.com\\
%% \hline
%% \end{tabular}
%% \caption{Authors}
%% \end{table}

\begin{table}[htb]
\centering
\begin{tabular}{ccc}
\hline
\bf Version & \bf date & \bf Update \\
\hline
\hline
0.1 & 30/09/2007  & Initial revision\\
\hline
\end{tabular}
\caption{Version history}
\end{table}


%\section{Introduction}


%\subsection{Purpose}

\begin{multicols}{2}


\section{Software description}

\subsection{Existent}

Amap is based on some open source code. Kmeans and hierarchical
clustering (with distance matrix computation) 
are taken from R-project. An R code implementing robust pca
described in \cite{caussinu+ruiz2} is provided by the article author.
Amap use Fortran code from Michel Petitjean \cite{mpetitjean}. 

A summary of evolutions:
\begin{itemize}
\item An evolution of Kmeans is required 
in order to satisfy exigence SRS\_3.
\item An evolution of hierarchical clustering is required in order
to satisfy exigence SRS\_4 and SRS\_6.
\item Robust pca must be written in C, C++ or Frotran in order
to satisfy exigence SRS\_5.
\item Interfaces of Fortran code from Michel Petitjean must
be reworked in order to fit with R call.
\end{itemize}


\subsection{Detailed description}

\req{Amap package use a modification of Kmeans function from R-project, to
provide a choice on distance used. Distance primitive function
are common between kmeans, distance (matrix computation) and hierachical
clustering}


\req{Amap package port hclust fortran function and distance C function
from R-project to C++ in order to
\begin{itemize}
 \item launch directly distance from an ``hcluster'' call; this will avoid duplication of matrix distance in R in order to satisfy SRS\_6.
 \item desalocate memory used by matrix distance after computation
 \item use many threads (number of threads will be a parameter sent by R command), in order to satisfy speed aspect of SRS\_6
 \item use templates in order to propose same algorithm in simple or double
floating point precision.
\item add new metrics, including pearson and rank based metrics (spearman).
 \end{itemize}
 ~
}

\req{Fortran code from Michel Petitjean is adapted to a use in R}
 
\req{A new code is developped in C to implement robust pca
described in \cite{caussinu+ruiz2}.}

\section{Validation}

\req{Validation of implementation of k-means, hierarchical clustering and
distance computation are done by two ways:
\begin{itemize}
\item  verification that r-project k-means gives same result when using
same metrics.
\item verification for all metrics that it computes the correct result,
with small examples (by hand).
\end{itemize}
Then results are archived to check non regression.
}

\req{Validation of implementation of robust pca is done by comparing
results from C code, and R code given by the author of article. Result
is now archived for non regression tests.}

\req{Validation of implementation of optimal partition will be done
by cheking that amap output and Michel Petitjean software output
are the same.
 Result
is now archived for non regression tests.}


\section{Methods, tools and techniques}



\req{C and C++ code will use doxygen inline documentation}

\req{Amap package will be host on a sourceforge-like project 
manager\footnote{At address http://mulcyber.toulouse.inra.fr/projects/amap}
i.e. a bug manager, a repository like
CVS and news for all releases or bug correction.}

\req{Amap will be compilant with SRS\_7}

\section{Tracability matrix}

%See Tab. \ref{matrix_srs_sdd}.

%\begin{table}[hbt]
%\label{matrix_srs_sdd}
\centering
\begin{tabular}{ccc}
\hline
\bf SRS & \bf SDD  \\
\hline
\hline
SRS\_1 & SDD\_4\\
SRS\_2 & SDD\_3\\
SRS\_3 & SDD\_1\\
SRS\_4 & SDD\_2\\
SRS\_5 & SDD\_1 SDD\_2 SDD\_3 SDD\_4\\
SRS\_6 & SDD\_2\\
SRS\_7 & SDD\_10\\
SRS\_8 & SDD\_5 SDD\_6 SDD\_7\\
SRS\_9 & SDD\_7\\
SRS\_10 &  SDD\_5 SDD\_6 SDD\_7\\
\hline
\end{tabular}
%\caption{Tracability matrix}
~\\

Tab.~2: Tracability matrix
%\end{table}




\bibliography{amap}

\end{multicols}

\end{document}
